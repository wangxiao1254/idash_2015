\documentclass[t,ignorenonframetext]{beamer}
\mode<presentation> {
  \usetheme{PaloAlto}
  %\usecolortheme{seagull}
  %\usefonttheme{serif}
}

\usefonttheme{structuresmallcapsserif}
\usepackage{enumerate,graphicx}
\usepackage[]{textpos}
\usepackage{comment}
\usepackage{algorithm}
\usepackage{caption}
\usepackage[noend]{algorithmic}
%\usepackage{vplace}
\DeclareCaptionFormat{myformat}{#3}
\captionsetup[algorithm]{format=myformat}
%\usepackage{algpseudocode}
%\usepackage{algorithmicx,color,xspace,xcolor,fullpage,url}
\setlength{\TPHorizModule}{5mm}
\setlength{\TPVertModule}{\TPHorizModule}
% The line below is what I talked about that makes all
% items in a list into overlays
%\beamerdefaultoverlayspecification{<+->}
\newcommand{\tc}[1]{$\backslash$\texttt{#1}}
\newcommand{\kartik}[1]{{\footnotesize\color{blue}[Kartik: #1]}}
\newcommand\myblock[4]{%
\begin{textblock}{#1}(#2, #3)%
  \begin{center}
    #4
  \end{center}
\end{textblock}
}
\newcommand\leftblock[4]{%
\begin{textblock}{#1}(#2, #3)%
  \vspace*{\fill}
    #4
  \vspace*{\fill}
\end{textblock}
}
%\algtext*{EndWhile}% Remove "end while" text
%\algtext*{ENDIF}% Remove "end if" text

\title[]{iDASH Secure Genome Analysis Competition Using ObliVM}
\author[]{Xiao Shaun Wang, Chang Liu, \textbf{Kartik Nayak}, Yan Huang and Elaine Shi}
\date{University of Maryland, College Park}
\begin{document}
\frame{
\maketitle
}
\section{Task 1a}

\begin{comment}
\begin{frame}
\begin{itemize}
\item Compute minor allele frequencies
\myblock{8}{0}{1}{Alice\\
$l^A = (e^A_1,...,e^A_n)$
}
\myblock{8}{8}{1}{Bob\\
$l^B = (e^B_1,...,e^B_n)$
}
\pause
\myblock{16}{0}{4}{where $e_i^X \in \{A, T, C, G\}$}
\pause
\myblock{8}{0}{6}{
AA AC AA\\
$f_1^A = 5, f_2^A = 1$
}
\myblock{8}{8}{6}{
AA AC CC\\
$f_1^B = 3, f_2^B = 3$
}
\pause
\myblock{16}{0}{9}{
Secure Computation\\
$(f_1, f_2) = (f^A_1+f^A_1, f^B_2+f^B_2)$\\
\pause
40 AND gates
}
\end{itemize}
\end{frame}

\section{Task 1B}
\label{sec:task1b}

\begin{frame}{Problem Statement}
  \begin{itemize}
    \item Task 1b: Computing $\chi^2$ statistic
    \myblock{10}{0}{1}{Alice\\
      $l^A_{case} = (e^A_1,...,e^A_n)$\\
      $l^A_{control} = (e'^A_1,...,e'^A_n)$
    }
    \myblock{10}{10}{1}{Bob\\
      $l^B_{case} = (e^B_1,...,e^B_n)$\\
      $l^B_{control} = (e'^B_1,...,e'^B_n)$
    }
    \pause
    \leftblock{20}{0}{5}{
      $a, b$: frequencies of two alleles in $l_{case} = l^A_{case} || l^B_{case}$\\
      $c, d$: frequencies of two alleles in $l_{control} = l^A_{control} || l^B_{control}$
    }
    \pause
    \myblock{20}{0}{8}{
      $\chi^2 = n\times\frac{(ad-bc)^2}{rsgk}$\\
      where $r = a + b, s = c + d, g = a + c,$\\$k = b + d, n =  r + s$
    }
  \end{itemize}
\end{frame}

\begin{frame}{Solution: Computing $\chi^2$ statistic}
    \leftblock{20}{0}{1}{
      $a, b$: frequencies of two alleles in $l_{case} = l^A_{case} || l^B_{case}$\\
      $c, d$: frequencies of two alleles in $l_{control} = l^A_{control} || l^B_{control}$
    }
    \pause
  \vspace{12mm}
  \begin{itemize}
    \item Compute partial values: \\$a^A,b^A,c^A,d^A$ for Alice \\$a^B,b^B,c^B,d^B$ for Bob
      \pause
    \item In Secure Computation, compute $a = a^A + a^B$ \ldots
  \end{itemize}
  \myblock{20}{0}{1}{
    $\chi^2 = n\times\frac{(ad-bc)^2}{rsgk}$
  }
\end{frame}

\begin{frame}{Results: Computing $\chi^2$ statistic}
  \begin{itemize}
    \item Floating point computation \kartik{number of bits used}
    \item Absolute accuracy\\
      $1.11 \times 10^{-4}$ with 7763 gates\\
      $5.6 \times 10^{-8}$ with 14443 gates
  \end{itemize}
\end{frame}

\section{Set union}
\label{sec:setunion}

\begin{frame}{Building Block: Secure Set Union}
  \myblock{10}{0}{0}{Alice\\
      $S_A$
    }
    \myblock{10}{10}{0}{Bob\\
      $S_B$
    }
    \myblock{20}{0}{2.5}{Cardinality of the union of the sets i.e. $|S_A \cup S_B|$}
    \leftblock{20}{0}{6}{Strawman solution:}
    \vspace{30mm}
    \begin{algorithm}[H]
      \begin{algorithmic}[1]
        \STATE Sort the input array $d[]$ obliviously.
        \STATE $cnt = 1$
        \FOR {$i=0:len(d)-1$}
	  \IF{$d[i] \neq d[i+1]$}
		\STATE $cnt = cnt + 1$
         \ENDIF
        \ENDFOR
        \RETURN $cnt$
      \end{algorithmic}
      \caption{union(d[]: a list of elements) \kartik{xiao}}
    \end{algorithm}
\end{frame}

\begin{frame}{Set Union: Oblivious Merge}
  \begin{itemize}
    \item Local sort of $S_A$ and $S_B$
    \item Oblivious merge of sorted lists $(O(n \log n))$
    \item Compute set union in a single pass
  \end{itemize}
\end{frame}

\begin{frame}{Set Union: Bloom Filter}
  \begin{itemize}
    \item Used to check for existence of elements
    \item Approximate the size of the set $S$
      \myblock{20}{0}{2}{
        \LARGE{$|S| = \frac{\ln(1-\frac{X}{m})}{k\ln(1-1/m)}$}
      }
      \leftblock{20}{0}{6}{
        where\\ $X$: number of bits set,\\ $m$: number of bits in the bloom filter \\
        $k$: number of hash functions
      }
  \end{itemize}
\end{frame}

\begin{frame}{Set Union: Bloom Filter}
  \begin{itemize}
    \item Compute bloom filters locally
    \kartik{image for local computation}
    \item In secure computation, Compute bitwise OR and count number of 1's
    \item Compute $|S|$ in cleartext
  \end{itemize}
\end{frame}
\end{document}